%version 3.03.16



\documentclass[10pt,a4paper]{article}

%\usepackage{Fmed541}

%Pakete allgemein
\usepackage[utf8]{inputenc}

\usepackage[T1]{fontenc}
\usepackage{graphicx}          		% Einbindung von Grafiken
\usepackage[usenames,dvipsnames]{color} % farbige Texte auch im PDF
\usepackage{lastpage}         	 	% Seitenzahl von Gesamtzahl
\usepackage{fancyhdr}          		% Fuflzeilengestaltung
\usepackage{multirow}          		% mehrere Zeilen pro Tabellen-Zelle
\usepackage{sectsty}           		% ƒnderung der Abschnittsgestaltung
\usepackage{fancyvrb}          		% verbatim Texte
\usepackage{supertabular}      		% lange Tabellen mit Seitenumbruca
\usepackage[pdftex,pdfpagemode=UseNone,pdfstartview=FitH, pdfborder={0 0 0}, colorlinks=true, urlcolor=blue, linkcolor=darkgray]{hyperref}	% Sprungmarken und Referenzen im PDF
\usepackage{framed, color}     		% farbige Texthintergr¸nde
\usepackage{colortbl}          		% farbige Tabellen (-kˆpfe)
\usepackage{wallpaper}         		% Hintergrundbild auf erster Seite
\usepackage{ifthen}      		% Bedingte Kommandos ermˆglichen
\usepackage{listings}
\usepackage{courier}
\usepackage{array, ragged2e}
\usepackage{datetime}

%Formate allgemein
\setlength{\parindent}{0pt}
\setlength{\parskip}{12pt}
\setlength{\topmargin}{-1.5cm}
\setlength{\headsep}{1.2cm}
\setlength{\headheight}{20pt}
\setlength{\footskip}{1.2cm}
\setlength{\textheight}{25cm}
\setlength{\textwidth}{17cm}
\setlength{\evensidemargin}{1cm}
\setlength{\oddsidemargin}{-0.6cm}
\setlength{\fboxsep}{4pt}
\setcounter{secnumdepth}{6}
\setcounter{tocdepth}{3}
\renewcommand{\rmdefault}{ftg}\rmfamily
\sectionfont{\fontsize{18}{18}\selectfont}
\subsectionfont{\normalfont\fontsize{14}{14}\selectfont}
\subsubsectionfont{\normalfont}
\paragraphfont{\normalfont}
\subparagraphfont{\normalfont}
\pagestyle{fancy}
\clubpenalty 9000 \widowpenalty 9000 \displaywidowpenalty 9000

\pdfminorversion=3

%Schalter allgemein

%Kommandos allgemein
\newcommand{\PreserveBackslash}[1]{\let\temp=\\#1\let\\=\temp}
\newcommand{\vbackslash}{$\backslash$}
\newcommand{\magenta}{\color{magenta}}
\newcommand{\schwarz}{\color{black}}
\newcommand{\grau}{\color{Gray}}
\newcommand{\rot}{\color{Red}}
\newcommand{\blau}{\color{blue}}
\newcommand{\royalblue}{\color{RoyalBlue}}
\definecolor{shadecolor}{gray}{.85}
\definecolor{dunkelgrau}{gray}{0.65}
\definecolor{hellgrau}{gray}{0.85}
\definecolor{darkgray}{gray}{0.4}

%Spaltentypen allgemein
\newcolumntype{X}{>{\cellcolor{hellgrau}}l} %graue Trennzeile in GS-Tabelle
\newcolumntype{P}{>{\raggedright} p{3.5cm}} %Produkt in GS-Tabelle
\newcolumntype{H}{>{\raggedright} p{1.3cm}} %Hersteller in GS-Tabelle
\newcolumntype{G}{>{\raggedright} p{1.4cm}} %Gesch‰ftsfall in GS-Tabelle
\newcolumntype{I}{>{\raggedright} p{2.7cm}} %Interface in GS-Tabelle
\newcolumntype{K}{>{\PreserveBackslash{\raggedright}} p{6.0cm}} %Konfigurationspfad in GS-Tabelle
\newcolumntype{V}{>{\raggedright} p{1cm}} %Version in Versionstabelle
\newcolumntype{D}{>{\raggedright} p{2cm}} %Datum in Versionstabelle
\newcolumntype{S}{>{\raggedright} p{2.5cm}} %Status in Versionstabelle
\newcolumntype{v}{>{\PreserveBackslash{\RaggedRight}} p{9.7cm}} %ƒnderungen in Versionstabelle

%Kommando-ƒnderungen allgemein

\renewcommand\paragraph{\@startsection{paragraph}{4}{\z@}
                                    {3.25ex \@plus1ex \@minus.2ex}
                                    {0.5ex \@plus .2ex}
                                    {\normalfont\normalsize}}

\renewcommand\subparagraph{\@startsection{subparagraph}{5}{\parindent}
                                       {3.25ex \@plus1ex \@minus .2ex}
                                       {0.1ex \@plus .2ex}
                                      {\normalfont\normalsize}}


%List Mono Font
%\usepackage[T1]{fontenc}
%\usepackage[scaled]{beramono}
%\newcommand\Small{\fontsize{9}{9.2}\selectfont}
%\newcommand*\LSTfont{\Small\ttfamily\SetTracking{encoding=*}{-60}\lsstyle}
%Listingformat allgemein
%% \lstset{language=,frame=tbrl,extendedchars=true,basicstyle=\fontsize{7}{7}\selectfont}
%% \lstset{language=,frame=tbrl,extendedchars=true,basicstyle=\footnotesize\ttfamily,columns=fullflexible,breaklines=true,breakatwhitespace=true,aboveskip=\bigskipamount}
%\lstset{language=,frame=tbrl,extendedchars=true,basicstyle=\LSTfont,columns=fixed,breaklines=true,breakatwhitespace=true,aboveskip=\bigskipamount}

\lstset{language=,frame=tbrl,extendedchars=true,basicstyle=\scriptsize\ttfamily,breaklines=true,breakatwhitespace=true,aboveskip=\bigskipamount,escapeinside={(*@}{@*)}}

%Kopfzeile allgemein
\renewcommand{\headrulewidth}{0pt}
\rhead{\fontsize{10}{10}\selectfont \testCaseId - \testCaseTitle \\[0pt]}
\lhead{\fontsize{10}{10}\selectfont Deutsche Telekom Technik GmbH \\[0pt]\rule{170mm}{0.1mm}~~~}
\chead{\fontsize{10}{10}\selectfont FMED-541\\[0pt]}

% Logo allgemein
%\newcommand{\TLogo}{\fontfamily{ftl}\fontsize{36}{36}\selectfont
%{\magenta ============================
%\hspace{-0.65ex}T\hspace{-0.65ex}===}\\}


%Fusszeile allgemein
\newcommand{\Fusszeile}
{
\fancyfoot[L]{\fontsize{10}{10}\selectfont \jobname.pdf}
\fancyfoot[C]{}
\fancyfoot[R]{\fontsize{10}{10}\selectfont  \thepage \ of \pageref{LastPage} }
}


%
%
%
% end Fmed541.sty




\usepackage[ngermanb]{babel}
\usepackage[german]{isodate}
\usepackage[table]{xcolor}% http://ctan.org/pkg/xcolor

\usepackage{tabularx}

\usepackage{xpatch}

\newenvironment{testHeader}
    {%
    \expandafter\xpretocmd\csname endtabular*\endcsname{\hline}{}{}%
    \tabularx{\linewidth}{|  c  |  X  |}
        \hline
        \cellcolor{gray!50} Testcase Id & \cellcolor{gray!50} Testcase Title \\
        \hline}
    {\endtabularx}
    
\newenvironment{testExecution}
    {%
    \expandafter\xpretocmd\csname endtabular*\endcsname{\hline}{}{}%
    \tabularx{\linewidth}{| X |  p{0.6cm} |  p{0.5cm} |  p{0.5cm}  | p{0.5cm} |}}
    {\endtabularx}

\newenvironment{testExecutionHeader}
    {%
    \expandafter\xpretocmd\csname endtabular*\endcsname{\hline}{}{}%
    \tabularx{\linewidth}{| X |  p{0.6cm} | p{0.5cm} | p{0.5cm} | p{0.5cm} |}
        \hline
        \cellcolor{gray!25} Test execution & \cellcolor{gray!25} >> & \cellcolor{gray!25} pass & \cellcolor{gray!25} fail & \cellcolor{gray!25} skip \\
        \hline}
    {\endtabularx}

\newenvironment{loopTable}
    {%
    \expandafter\xpretocmd\csname endtabular*\endcsname{\hline}{}{}%
    \tabularx{\linewidth}{|  c  |  c  | X  |}
        \hline
        \cellcolor{gray!50} loop & \cellcolor{gray!50} check name & \cellcolor{gray!50} textfsmOutput \\
        \hline}
    {\endtabularx}

\usepackage{tikz}
\usetikzlibrary{calc,trees,positioning,arrows,chains,shapes.geometric,%
    decorations.pathreplacing,decorations.pathmorphing,shapes,%
    matrix,shapes.symbols,automata} 

\usepackage{newfloat}
\usepackage{hyperref}
\usepackage{sidecap}
\DeclareFloatingEnvironment[fileext=frm,placement={!ht},name=CLI Output]{cliOutput}
\usepackage{caption}
\captionsetup[cliOutput]{labelfont=bf,font=small,
  justification=raggedright,
  singlelinecheck=false
}    

\usepackage{attachfile}
\usepackage{fancyvrb}

\newboolean{skip}
\newboolean{pass}

\font\tt=rm-lmtl10
\font\itt=rm-lmtlo10
\font\btt=rm-lmtk10
\font\bitt=rm-lmtko10

\begin{document}
\sloppy

\newcommand{\testCaseId}{\VAR{configDict["name"]}}
\newcommand{\testCaseTitle}{\VAR{configDict["title"]}}
\newcommand{\testCaseCompound}{\testCaseId \testCaseTitle}
\newcommand{\testDescription}{\VAR{configDict["description"]|replace("_", "\_")|replace('"', "''")}}
\newcommand{\testExpected}{\VAR{configDict["expected"]|replace("_", "\_")|replace('"', "''")}}
\newcommand{\testPreparation}{\VAR{configDict["preparation"]|replace("_", "\_")|replace('"', "''")}}

\newcommand{\Teil}{\VAR{configDict["pdfOutput"]["title1"]}}
\newcommand{\TestCase}{\VAR{configDict["name"]} - \VAR{configDict["title"]}}

\newcommand{\Version}{1.0}
\newcommand{\Author}{\VAR{configDict["pdfOutput"]["author"]}}
\newcommand{\Tester}{\VAR{configDict["tester"]}}

%\newcommand{\Versionen}{\subsection{1.0t}}
%\newcommand{\WeitereMitgeltendeUnterlagen}{}

%TestResult allgemein
\newpage
%~\\
%%\vfill
\section { Test Report Overview }  

\begin{testHeader}
\cellcolor{gray!25} \testCaseId & \cellcolor{gray!25} \testCaseTitle   \\  \hline
Description     & \testDescription   \\  \hline
Expected        & \testExpected      \\  \hline
Preparation     & \testPreparation   \\  \hline
Author       &  \Author    \\  \hline
Tester       &  \Tester  \\  \hline
Date   & \today\ \currenttime   \\  \hline
\end{testHeader}


\begin{testExecutionHeader}
\end{testExecutionHeader}
\label{link-testExecution}
 \BLOCK{ for y in configDict["steps"] }
 \begin{testExecution} 
  \BLOCK{ for stepType,stepDict in y.items() }
    \BLOCK{ for x in stepDict["output"] }
      \BLOCK{ if x["skip"] }
         \setboolean{skip}{true} 
         \setboolean{pass}{false} 
       \BLOCK{ else }
        \BLOCK{ if x["pass"] }
         \setboolean{pass}{true} 
        \BLOCK{ else }
         \setboolean{pass}{false} 
        \BLOCK{ endif }
      \BLOCK{ endif }
    \BLOCK{ endfor }
   \ifthenelse{\boolean{skip}}
    {Step \VAR{stepDict["stepCount"]}: \VAR{stepDict["output"][0]["name"]} &  & - & - & \cellcolor{yellow!25} skip \\ \hline }
    {\ifthenelse{\boolean{pass}}
      {Step \VAR{stepDict["stepCount"]}: \VAR{stepDict["output"][0]["name"]} &  & \cellcolor{green!25} pass & - & - \\ \hline }
      {Step \VAR{stepDict["stepCount"]}: \VAR{stepDict["output"][0]["name"]}  &  & - & \cellcolor{red!25} fail & - \\ \hline }
    }
  \BLOCK{ if stepType == "check" or stepType == "record" or stepType == "config" or stepType == "comment" or stepType == "await"}
   \BLOCK{ for o in stepDict["output"] }
    \BLOCK{ if ( stepType == "check" or stepType == "record" or stepType == "await") and ( stepDict["hideOutput"] is not defined  ) }
\tt\tiny  \VAR{o["commandTt"]} & \tiny \ref{label-\VAR{o["stepCount"]}-\VAR{o["commandCount"]}} & " "  &  " " & " " \\
    \BLOCK{ else }
\tt\tiny  \VAR{o["commandTt"]} & " " & " "  &  " " & " " \\
    \BLOCK{ endif }
   \BLOCK{ endfor }
  \BLOCK{ endif }
   \hline
  \BLOCK{ endfor }
  \end{testExecution}
 \BLOCK{ endfor }

%\end{testExecution}

\section { Test Setup }  

\BLOCK{ if "tikzpicture" in configDict["pdfOutput"] }
\VAR{configDict["pdfOutput"]["tikzpicture"]}
\BLOCK{ endif }

\BLOCK{ if "tikzmatrix" in configDict["pdfOutput"] }
        \begin{tikzpicture}[
          font=\sffamily,
          every matrix/.style={ampersand replacement=\&,column sep=0cm,row sep=2cm},
          dut/.style={draw,thick,circle,fill=gray!40},
          router/.style={draw,thick,circle,fill=gray!20},
          system/.style={draw,thick,rounded corners,fill=gray!20,inner sep=.4cm},
          ip2/.style={draw,thick,rounded corners,fill=gray!20},
          interface/.style={draw,thick,rounded corners,fill=gray!20},
          lineId/.style={draw,thick,rounded corners,fill=gray!10},
          to/.style={->,>=stealth',shorten >=1pt,semithick,font=\sffamily\footnotesize},
          biDir/.style={<->,>=stealth',shorten >=1pt,semithick,font=\sffamily\footnotesize},
          every node/.style={align=center}]
\VAR{configDict["pdfOutput"]["tikzmatrix"]}
\end{tikzpicture}
\BLOCK{ endif }


\pagebreak


\section { Output }  
\BLOCK{ for y in configDict["steps"] }
 \BLOCK{ for stepType,stepDict in y.items() }
  \BLOCK{ if stepType == "check" or stepType == "record" or stepType == "comment" or stepType == "await" }
  \BLOCK{ if stepDict["hideOutput"] is not defined }
  \subsection {Step-\VAR{stepDict["stepCount"]} \VAR{stepDict["output"][0]["name"]} }
  \small \VAR{stepDict["description"]|replace("_", "\_")|replace('"', "''")}
  \BLOCK{ if stepType == "check" or stepType == "record" or stepType == "await" }
   \BLOCK{ for x in stepDict["output"] }
   \subsubsection { \tt Step-\VAR{x["stepCount"]}-Command-\VAR{x["commandCount"]}: \VAR{x["commandTt"]} }
   \label{label-\VAR{x["stepCount"]}-\VAR{x["commandCount"]}}

\rm\tiny{\underline{verbatim command}: \quad \verb+\VAR{x["commandOrig"]}+} \\
\rm\tiny{\underline{Device}: \VAR{stepDict["device"]}\\}
\rm\tiny{\underline{Step-Name}: \VAR{stepDict["name"]}\\}
\rm\tiny{\underline{TS(end)}: \VAR{x["endTS"]} \quad \underline{Elapsed}: \VAR{x["elapsed"]} \\}
%\tt\scriptsize{\begin{verbatim} \VAR{x["commandOrig"]} \end{verbatim} } } 

     \begin{lstlisting}[]]
\VAR{x["output"]}
     \end{lstlisting}
     \BLOCK{ if x["stepType"] == "check" or x["stepType"] == "await"}
       \VAR{x["stepType"]} pass result: \VAR{x["pass"]}
       \tiny \begin{lstlisting}[]]
\VAR{x["checkResult"]}
       \end{lstlisting}
     \BLOCK{ endif }
     \tt\scriptsize goto top:\ref{link-testExecution} \normalsize \\
   \BLOCK{ endfor }
  \BLOCK{ endif }
  \BLOCK{ if stepType == "comment"}
   \BLOCK{ for x in stepDict["output"] }
      \BLOCK{ if x["lstlisting"]|length > 1 }
       \begin{lstlisting}[]]
\VAR{x["lstlisting"]}
       \end{lstlisting}
      \BLOCK{ endif }
      \BLOCK{ if x["inludeImageFromFile"]|length > 1 }
\graphicspath{{../../}}
\includegraphics[width=1\textwidth]{\VAR{x["inludeImageFromFile"]}}
      \BLOCK{ endif }
      \BLOCK{ if x["attachFile"]|length > 1 }
\graphicspath{{../../}}
\attachfile[icon=Paperclip]{\VAR{x["attachFile"]}}
      \BLOCK{ endif }
   \BLOCK{ endfor }
  \BLOCK{ endif }
  \BLOCK{ endif }
  \BLOCK{ endif }
 \BLOCK{ endfor }
\BLOCK{ endfor }




\pagebreak
\section { Attachments }  

\subsection {yaml runbook}  
\tiny \begin{verbatim}\VAR{configDict["configFile"]}\end{verbatim} \normalsize
\attachfile[icon=Paperclip]{\VAR{configDict["configFile"]}}

%\subsection { latex source code }  
%\attachfile[icon=Paperclip]{\jobname.tex}

\BLOCK{ if "appendFiles" in configDict["pdfOutput"] }
 \BLOCK{ for appendFile in configDict["pdfOutput"]["appendFiles"] }
    \subsection {\VAR{appendFile|replace("_", " ")|replace('"', "")}}
    \attachfile[icon=Paperclip]{\VAR{appendFile}}
 \BLOCK{ endfor }
\BLOCK{ endif }

\end{document}